\documentclass[12pt]{article}
\usepackage{times}
\usepackage[english]{babel}
\usepackage[utf8]{inputenc}
\usepackage{amsmath}
\usepackage{graphicx}
\usepackage[colorinlistoftodos]{todonotes}
\usepackage{setspace}
\usepackage{pdfpages}
\usepackage{url}
\usepackage{caption}
\usepackage{subcaption}
\captionsetup[figure]{font=small,labelfont=normalsize}
\usepackage[titletoc]{appendix}
\onehalfspace
\usepackage{geometry}
	\geometry{
	a4paper,
	left = 30mm,
	right = 30mm,
	top = 25mm,
	bottom = 20mm
	}

\usepackage[export]{adjustbox}
\makeatletter
\setlength{\@fptop}{0pt}
\makeatother

\begin{document}
\section{Acoustic Contrast Control}
Acoustic contrast control was first proposed by Choi and Kim \cite{choi2002generation} as a constrained optimization cost function that maximizing the ratio between the mean square pressure in a bright (listening) and dark (quite) zone, denoted as the subscripts $b$ and $d$. Acoustic contrast (AC) is defined as 
\begin{equation}
AC = 10log_{10} \bigg (\frac{L_d \mathbf{p}_{b}^{H}\mathbf{p}_{b}}{L_b\mathbf{p}_{d}^{H}\mathbf{p}_{d}} \bigg) = 10log_{10} \bigg (\frac{L_d \mathbf{q}^{H}\mathbf{G}_{b}^{H}\mathbf{G}_{b}\mathbf{q}}{L_b\mathbf{q}^{H}\mathbf{G}_{d}^{H}\mathbf{G}_{d}\mathbf{q}} \bigg),
\end{equation}
where $\mathbf{p}$ is an $L$ x $1$ column vector of pressure, $\mathbf{q}$ is an $M$ x $1$ column vector of complex source strengths, $\mathbf{G}$ is an $L$ x $M$ matrix of acoustic transfer functions between each $M$ control sources and $L$ control points, and $H$ denotes the Hermitian operator. The AC is maximized by solving a constrained cost function where $\mathbf{p}_{b}^{H} \mathbf{p}_{b}$ is maximized under the constraint that  $\mathbf{p}_{d}^{H} \mathbf{p}_{d}$ is kept at a constant real value $D$. With the use of the Lagrangian multiplier method, the cost function to be maximized is given by
\begin{equation}
J(\mathbf{q},\lambda) =\mathbf{q}^{H}\mathbf{G}_{b}^{H}\mathbf{G}_{b}\mathbf{q} - \lambda(\mathbf{q}^{H}\mathbf{G}_{d}^{H}\mathbf{G}_{d}\mathbf{q} - D),
\end{equation}
where $\lambda$ is the Lagrangian multiplier. Taking the partial derivative with respect to $\mathbf{q}$ and $\lambda$ and equating the results to $0$, yields the stationary points as 
\begin{equation}
\lambda \mathbf{q} = [ \mathbf{G}_{d}^{H} \mathbf{G}_{d} ]^{-1} [ \mathbf{G}_{b}^{H} \mathbf{G}_{b} ] \mathbf{q},
\end{equation}
\begin{equation}
\mathbf{q}^{H}\mathbf{G}_{d}^{H}\mathbf{G}_{d}\mathbf{q} = D.
\label{lamb}
\end{equation}
 It was shown by Choi and Kim \cite{choi2002generation} that the optimal source strengths $\mathbf{q}$ that maximizes the AC is proportional to the eigenvector that corresponds to the max eigenvalue of $[ \mathbf{G}_{d}^{H} \mathbf{G}_{d} ]^{-1} [ \mathbf{G}_{b}^{H} \mathbf{G}_{b} ] $. $\lambda$ is then chosen manually so that Equation \ref{lamb} is met.






\raggedright
\bibliographystyle{IEEEtran}
\bibliography{References}
\small


\end{document}





